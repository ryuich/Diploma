%%%
%%%  VZOR PRO VYTVOŘENÍ DIPLOMOVÉ PRÁCE 
%%%  
%%%  * soubor obsahující fiktivní první kapitolu
%%%
%%%  AUTOŘI:  Arnošt Komárek (komarek@karlin.mff.cuni.cz), 2011
%%%           Michal Kulich (kulich@karlin.mff.cuni.cz), 2013
%%%
%%%  POSLEDNÍ ÚPRAVA: 20130315
%%%  
%%%  ===========================================================================

\chapter{Základní pojmy}

Nechť $\left( \Omega, \mathcal{F}, \mathbb{P} \right)$ je úplný
pravděpodobnostní prostor. Více na thema frakcionálního Brownova pohybu např.
\cite{nourdin2013selected}.


\begin{definice}\label{def:fbm}
    Standardním frakcionálním Brownovým pohybem (\vspace{2pt}fBm) indexovaný
    Hurstovým parametrem $H\in\left(0,1\right)$ nazveme spojitý centrovaný
    gaussovský proces $\left( B^H (t),t\geq 0\right)$ s kovarianční funkcí
    \begin{align*}
      \mathbb{E}& \left[ B^H (s) \, B^H (t) \right] =
      \frac{1}{2}\left( t^{2H}+s^{2H}-|t-s|^{2H} \right).
    \end{align*}
\end{definice}


\begin{definice}
    Náhodný proces $\left(B^H(t),t\geq0\right)$ nazveme cylindrickým frakcionálním Brownovým
    pohybem s Hurstovým parametrem $H\in\left(1/2,1\right)$ na prostoru $V$, definovaný na
    filtrovaném pravděpodobnostním prostoru $\left( \Omega, \mathscr{F}, \left(
    \mathscr{F}_t\right)_{t\geq0},\mathbb{P} \right)$, pokud
    
    $$ B^H(t) = \sum_{i=1}^\infty e_i \sqrt{\lambda_i}\, \beta_i (t), \qquad
    t>0,$$
    kde $\left\{ e_i, i\in\mathbb{N} \right\}$ je úplná ortonormální báze
    prostoru $V$, $\left( \beta_i(t),i\in\mathbb{N},t\geq0 \right)$ je rodina
    (stochasticky) nezávislých, reálných standardních frakcionálních
    Brownových pohybů s pevně zvoleným Hurstovým parametrem $H\in\left(
    {\frac{1}{2},1}\right)$ a $\lambda\geq0$, $\left(
    \lambda_i,i\in\mathbb{N} \right)$ je omezená poslopnoust v
    $\mathbb{R}_+$.
\end{definice}


\begin{definice}
    Řekneme, že filtrace $\left( \mathscr{F}_t,t\geq0 \right)$ splňuje obvyklé podmínky, pokud
    \begin{enumerate}[(i)]
        \item $\mathscr{F}_t$ je $\mathbb{P}$-úplná,
        \item $\mathscr{F}_t = \bigcap_{s>t} \mathscr{F}_s$  (zprava spojitá).
    \end{enumerate}
\end{definice}


\begin{definice}
    Inkrementální kovariance $\tilde{Q}$ cylindrického fBm $\left(B^H(t),t\geq0\right)$
    je definovaná vztahem
    $$ \tilde{Q}e_n = \lambda_n e_n, \qquad n\in\mathbb{N}. $$ 
\end{definice}


\begin{definice}[\cite{duncan2013linear}]\label{def:fracint}
    Nechť $\alpha\in \left( 0,1 \right)$ je pevné. Potom levostranný, popř.
    pravostranný frakcionální (Riemann--Liouvilleův) integrál pro
    $\varphi\in L^1 \left( \left[ 0,T \right] \right)$ definovaný pro skoro
    všechna $t\in\left[ 0,T \right]$ je dán předpisem
        
        $$\left( I_{0_+}^\alpha \varphi \right) \left( t \right) \eqdef
        \frac{1}{\Gamma(\alpha)}\int_{0}^{t} \left(t-s\right)^{\alpha-1} \varphi(s)
        \dd s,$$
    popř.
        $$\left( I_{T_-}^\alpha \varphi \right) \left( t \right) \eqdef
        \frac{1}{\Gamma(\alpha)}\int_{t}^{T} \left(s-t\right)^{\alpha-1} \varphi(s)
        \dd s,$$
        kde $\Gamma(\alpha)=\int_0^\infty x^{\alpha-1}e^{-x}\dd x$ je gama funkce.
\end{definice}


\begin{definice}[\cite{duncan2013linear}]\label{def:fracder}
    Nechť $\alpha\in \left( 0,1 \right)$ je pevné. Potom levostranná, popř.
    pravostranná frakcionální (Riemann--Liouvilleva) derivace pro
    $\varphi\in L^1 \left( \left[ 0,T \right] \right)$ definovaná pro skoro
    všechna $t\in\left[ 0,T \right]$ je dána předpisem
        
        $$\left( D_{0_+}^\alpha \varphi \right) \left( t \right) \eqdef
        \frac{\mathrm{d}}{\mathrm{d}t}\left( I_{0_+}^{1-\alpha} \varphi \right) \left( t \right) ,$$
    popř.
        $$\left( D_{T_-}^\alpha \varphi \right) \left( t \right) \eqdef
        \frac{\mathrm{d}}{\mathrm{d}t}\left( I_{T_-}^{1-\alpha} \varphi \right) \left( t \right) ,$$
    což ve Weylově reprezentaci dostáváme
        %$$\left( D_{0_+}^\alpha \varphi \right) \left( t \right) =
        %\frac{1}{\Gamma(1-\alpha)}\left[ \frac{\varphi(t)}{t^\alpha} + \alpha
            %\int_{0}^{t} \frac{\varphi(t)-\varphi(s)}{\left(t-s\right)^{\alpha+1}} 
        %\dd s, \right],$$
    %popř.
        %$$\left( D_{T_-}^\alpha \varphi \right) \left( t \right) =
        %\frac{1}{\Gamma(1-\alpha)}\left[ \frac{\varphi(t)}{\left( T-t
        %\right)^\alpha} + \alpha \int_{t}^{T} \frac{\varphi(s)-\varphi(t)}{\left(s-t\right)^{\alpha+1}} 
        %\dd s \right].$$
\begin{align*}
    \left( D_{0_+}^\alpha \varphi \right) \left( t \right) &=
        \frac{1}{\Gamma(1-\alpha)}\left[ \frac{\varphi(t)}{t^\alpha} + \alpha
        \int_{0}^{t} \frac{\varphi(t)-\varphi(s)}{\left(t-s\right)^{\alpha+1}} 
        \dd s, \right],\\
    \intertext{popř.}
    \left( D_{T_-}^\alpha \varphi \right) \left( t \right) &=
        \frac{1}{\Gamma(1-\alpha)}\left[ \frac{\varphi(t)}{\left( T-t
        \right)^\alpha} + \alpha \int_{t}^{T} \frac{\varphi(s)-\varphi(t)}{\left(s-t\right)^{\alpha+1}} 
        \dd s \right].
\end{align*}
\end{definice}


%\begin{definice}
    %Formální derivaci (\vspace{2pt}fBm) nazýváme frakcionálním gaussovským šumem (fGn).
%\end{definice}


\begin{definice}
    Prostorem $L^2_H \left( \left[ 0,T \right] \right)$ nazveme hilbertův
    prostor, kde $f,g\in L^2_H$, pokud $\langle f,f \rangle_H < \infty$ a
    $\langle g,g \rangle_H < \infty$ a skalární součin je dán jako

    $$ \langle f,g \rangle_H = \rho (H) \int_0^T u_{\frac{1}{2}-H}^2 (r)
    \left(I_{T^-}^{\left(H-\frac{1}{2}\right)} u_{H-\frac{1}{2}}\; f \right) (r)
    \left(I_{T^-}^{\left(H-\frac{1}{2}\right)} u_{H-\frac{1}{2}}\; g \right)
    (r) \dd r, $$
    kde 
    $$ \rho (H) = \frac{2H \, \Gamma (H+\frac{1}{2}) \, \Gamma(\frac{3}{2}-H)}
    {\Gamma (2-2H)}$$
    a $u_a (s) = s^a$, $a>0$, $s>0$.

\end{definice}


\begin{tvrz}[\cite{duncan2006prediction}]
    Nechť $f\in L^2_H \left( \left[ 0,T \right] \right)$. Potom
        $$\mathbb{E}\left[\int_s^t f \dd B^H \;|\; B^H (r), r\in \left(s,t\right)\right]
        = \int_0^s u_{\frac{1}{2}-H} \left(I_{s^-}^{-\left(H-\frac{1}{2}\right)} 
        \left(I_{t^-}^{\left(H-\frac{1}{2}\right)} u_{H-\frac{1}{2}}\; f
        \right) \right) \dd B^H, $$
    kde $u_a (s) = s^a$, $a>0$, $s>0$.
\end{tvrz}


%%%  ===========================================================================
