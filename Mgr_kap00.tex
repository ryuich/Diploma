%%%
%%%  VZOR PRO VYTVOŘENÍ DIPLOMOVÉ PRÁCE 
%%%  
%%%  * soubor obsahující fiktivní první kapitolu
%%%
%%%  AUTOŘI:  Arnošt Komárek (komarek@karlin.mff.cuni.cz), 2011
%%%           Michal Kulich (kulich@karlin.mff.cuni.cz), 2013
%%%
%%%  POSLEDNÍ ÚPRAVA: 20130315
%%%  
%%% =========================================================================

\chapter*{Úvod}
\addcontentsline{toc}{chapter}{Úvod}

\epigraph{\textit{``Begin at the beginning," the King said gravely, ``and go on
till you come to the end: then stop."}}{--- \textup{Lewis Carroll},\\ \textit{
Alice in Wonderland}}

Zde v bodech je shrnutí plánovaného členění této práce:
\begin{itemize}
    \item V první kapitole budou zavedeny pojmy, které se nadále budou používat
        v textu.
    \item Ve druhé kapitole budou shrnuty výsledky, ke kterým se již došlo v
        oblasti optimálního řízení rovnic s gaussovským šumem, se zaměřením na
        frakcionální Brownův pohyb.
    \item Ve třetí kapitole bude prostor pro případné důležité příklady a
        případné využití optimálního řízení ve finanční matematice.
    \item Čtvrtá kapitola by mohla nabídnout případné numerické porovnání výsledků
        dosahovaných při použití modelů s Brownovým pohybem a frakcionálním
        Brownovým pohybem a empirickými daty.
\end{itemize}

%Frakcionální Brownův pohyb představuje rodinu centrovaných gaussovských
%procesů, která je indexovaná Hurstovým parametrem $H\in\left(0,1\right)$.
%\vspace*{\baselineskip}

% \section*{Povídání}
%\addcontentsline{toc}{section}{Povídání}

%Řízení lineárních stochastických systémů s bílým šumem a hodnotovou funkcí
%kvadratickou v stavové proměnné (což se často nazývá LQG ({\em linear-quadratic
%Gaussian}) control problem) je pravděpodobně nejznámějším problémem
%stochastického optimálního řízení ve spojitém čase.

%Nyní bych rád popsal stávající situaci výzkumu:

Kleptsyna, LeBreton a Viot v pracích
\cite{kleptsyna2003linear,kleptsyna2005infinite} obdrželi optimální řízení pro
skalární stochastický systém s {\em fBm} pro $H\in\left(\frac{1}{2},1\right)$.
Pro případ částečné pozorovatelnosti navazují předem zmínění autoři prací
\cite{kleptsyna2008separation}. Hu a Zhou \cite{hu2005stochastic} odvodili
optimální řízení pro skalární bilineární systém s {\em fBm} pro
$H\in\left(\frac{1}{2},1\right)$ za podmínky, že takovéto řízení je markovské.
Duncan a Pasik-Duncan popsali některé speciální případy vícerozměrného systému
v \cite{duncan2009control} a \cite{duncan2010stochastic}. Optimální řízení
linearních rovnic v Hilbertově prostoru s {\em fBm} pro
$H\in\left(\frac{1}{2},1\right)$ a kvadratický hodnotový funkcionál explicitně
vyřešili Duncan, Maslowski a Pasik-Duncan v \cite{duncan2012linear}. Dále
Duncan a Pasik-Duncan v \cite{duncan2013linear} se zabývali optimálním řízením
lineárních stochastických systémů s libovolným centrovaným, kvadraticky
integrovatelným stochastickým procesem se spojitými trajektoriemi a hodnotovým
funkcionálem kvadratickým ve stavové proměnné, s explicitím vyjádřením případu
{\em fBm}, kde využili zobecněné metody doplnění na čtverec.


%%%%% ==============================================================================
