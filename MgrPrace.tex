%%%
%%% ŠABLONA PRO DIPLOMOVOU PRÁCI MFF UK - MATEMATIKA
%%%  
%%%  * hlavní soubor (Masterfile)
%%%
%%%  Tato šablona předpokládá kompilaci souboru pomocí sekvence:
%%%    cslatex -> bibtex -> cslatex (2x) -> dvips -> ps2pdf
%%%  Pro použití s latexem, pdflatexem a pdfcslatexem je potřeba
%%%  některé části trochu upravit.
%%%
%%%  AUTOŘI:  Martin Mareš (mares@kam.mff.cuni.cz)
%%%           Arnošt Komárek (komarek@karlin.mff.cuni.cz), 2011
%%%           Michal Kulich (kulich@karlin.mff.cuni.cz), 2013
%%%
%%%  POSLEDNÍ ÚPRAVA: 20130315
%%%  
%%%  ===========================================================================

%%%%% Základní nastavení pro jednostranný tisk:
%%%%% ----------------------------------------------------
% Okraje: levý 40mm, pravý 25mm, horní a dolní 25mm (ale pozor, LaTeX si sám přidává 1in)
\documentclass[12pt, a4paper]{report}
\setlength\textwidth{145mm}
\setlength\textheight{247mm}
\setlength\oddsidemargin{15mm}
\setlength\evensidemargin{15mm}
\setlength\topmargin{0mm}
\setlength\headsep{0mm}
\setlength\headheight{0mm}
% \openright zařídí, aby následující text začínal na pravé straně knihy
\let\openright=\clearpage


%%%%% Základní nastavení pro oboustranný tisk:
%%%%% ----------------------------------------------------
% \documentclass[12pt, a4paper, twoside, openright]{report}
% \setlength\textwidth{145mm}
% \setlength\textheight{247mm}
% \setlength\oddsidemargin{15mm}
% \setlength\evensidemargin{0mm}
% \setlength\topmargin{0mm}
% \setlength\headsep{0mm}
% \setlength\headheight{0mm}
% \let\openright=\cleardoublepage


%%%%% Nastavení kódování vstupních souborů: UTF-8
%%%%% ---------------------------------------------------------------
\usepackage[utf8]{inputenc} 



%%%%% Nastavení češtiny (slovenština analogicky)
%%%%% ---------------------------------------------------------------

%%% Existují dvě hlavní možnosti, jak zacházet s češtinou. Je zapotřebí zvolit právě jednu.
%%%

%%% MOŽNOST 1 (doporučujeme):
%%% * použití balíčku czech
%%%   (mimo jiné již obsahuje příkaz \uv pro sazbu českých uvozovek)
%%% * kompilace musí následně probíhat pomocí CSLaTeXu (příkaz
%%%   cslatex, resp. cspdflatex)
\usepackage{czech}

%%% MOŽNOST 2: (zde zakomentovaná)
%%% * použití balíčku babel s volbou pro češtinu
%%% * kompilace následně probíhá standardním LaTeXem (příkaz latex,
%%% resp. pdflatex)
% \usepackage[czech]{babel}
% \ifx\uv\undefined\newcommand{\uv}[1]{,,#1``}\fi     
%%% příkaz pro sazbu českých/slovenských uvozovek
%%% (v novějších verzích babelu je již k dispozici, stejně tak je již
%%% k dispozici v balíčku czech) 


%%% Další užitečné balíčky (jsou součástí běžných distribucí LaTeXu)
%%% ----------------------------------------------------------------
\usepackage{amsmath}        %%% rozšíření pro sazbu matematiky
\usepackage{amsfonts}       %%% matematické fonty
\usepackage{amssymb}        %%% matematické symboly
\usepackage{eucal}          %%% matematické fonty
\usepackage{mathrsfs}
\usepackage{amsthm}         %%% sazba vět, definic apod.
\usepackage{bm}             %%% tučné symboly (příkaz \bm)
\usepackage{graphicx}       %%% vkládání obrázků
\usepackage{psfrag}         %%% dodatečná úprava popisků v postscriptových obrázcích
\usepackage{fancyvrb}       %%% vylepšené prostředí pro strojové písmo
%\usepackage{natbib}         %%% zajištuje možnost odkazovat na
                            %%% reference stylem AUTOR (ROK), resp.
                            %%% AUTOR [ČÍSLO]  
\usepackage{cite}           %%% zajištuje možnost odkazovat se
\usepackage{bbding}         %%% balíček s nejrůznějšími
                            %%% symboly (čtverečky, hvězdičky,
                            %%% tužtičky, ručičky, nůžtičky, ...) 

% \usepackage{icomma}         %%% inteligetní čárka v matematickém módu
\usepackage{dcolumn}        %%% lepší zarovnání sloupců v tabulkách
\usepackage{booktabs}       %%% lepší vodorovné linky v tabulkách
\usepackage{paralist}       %%% lepší enumerate a itemize 
\usepackage{indentfirst}    %%% zaveď odsazení 1. odstavce
                            %%% kapitoly (v češtině se tyto
                            %%% odstavce odsazují) 
\usepackage[nottoc]{tocbibind} %%% zajistí přidání seznamu literatury,
                              %%% obrázků a tabulek do obsahu
\usepackage{lmodern}        %%% místo computer Modern se budou používat fonty
                            %%% Latin Modern; fonty ze skupiny IL2 už nedělají
                            %%% žadné potíže, fonty ze skupiny OMS stále ano

%%% hyperref: zajištuje generování hyperodkazů, bookmarků atp.
%%%     * předefinovává mnoho příkazů, měl by být proto uveden jako
%%%     poslední mezi seznamem zahrnutých balíčků        
%%%     * v ukázce níže jsou přidána některá nastavení, která lze
%%%     měnit dle libosti 
\usepackage[unicode]{hyperref}
\hypersetup{pdftitle=Optimální řízení stochastických lineárních rovnic s gaussovským šumem,
            pdfauthor=Jan Poslušný,
            ps2pdf,
            colorlinks=false,               %% hyperlinky budou označeny červenými rámečky, které budou neviditelné při tisku na papír
            urlcolor=blue,
            pdfstartview=FitH,
            pdfpagemode=UseOutlines,
            pdfnewwindow,
            breaklinks                      %% zajistí, aby se dlouhé hyperodkazy mohly lámat přes více řádků
}



%%% Příkazy zjednodušující přenositelnost
%%% -------------------------------------
\newcommand{\FIGDIR}{./Obrazky}    %%% cesta do adresare s obrazky

\makeatletter
\def\thmhead@plain#1#2#3{%
  \thmname{#1}\thmnumber{\@ifnotempty{#1}{ }\@upn{#2}}%
  \thmnote{ {\the\thm@notefont#3}}}
\let\thmhead\thmhead@plain
\makeatother


%%% Zavedení definic, vět, tvrzení, příkladů...
%%% vyžaduje balíček amsthm
\theoremstyle{plain}
\newtheorem{veta}{Věta}[chapter]                %% [chapter] prida cislovani podle kapitol
\newtheorem{lemma}[veta]{Lemma}
\newtheorem{tvrz}[veta]{Tvrzení}

\theoremstyle{plain}
\newtheorem{definice}{Definice}[chapter]

\theoremstyle{remark}
\newtheorem*{dusl}{Důsledek}
\newtheorem*{pozn}{Poznámka}
\newtheorem*{prikl}{Příklad}


%%% Prostředí pro důkazy zavedeme zvlášť
%%% Vyžaduje balíček bbding
%%% ------------------------------------

\newenvironment{dukaz}{
  \par\medskip\noindent
  \textit{Důkaz}.
}{
\newline
\rightline{\SquareCastShadowBottomRight}
}


%%% Seznam použité literatury
%%% Příkaz \bibliographystyle určuje, jakým stylem budou citovány
%%% odkazy v textu, a podle jakého stylu se automaticky vygeneruje
%%% seznam literatury. V závorce je název zvoleného .bst souboru.
%%% Styly plainnat a unsrt jsou standardní součástí latexových
%%% distribucí. Styl czplainnat vyžaduje přítomnost souboru
%%% czplainnat.bst ve stejném direktoráři, v němž se nachází
%%% kompilovaná práce. 
%%%
%%% Seznam literatury se vytváří na konci práce příkazem \bibliography, kde v závorce
%%% uvádíme název databázového bib souboru. 
%%% 
%%%
%\bibliographystyle{czplainnat}    %% Autor (rok) s českými spojkami
%\bibliographystyle{plainnat}     %% Autor (rok) s anglickými spojkami
%\bibliographystyle{unsrt}        %% [číslo]
%\bibliographystyle{plain}        %% [číslo]
\bibliographystyle{siam}        %% [číslo]

\renewcommand{\bibname}{Seznam použité literatury}


%%%%% Použití fancyvrb (fancy verbatim) při definici prostředí pro
%%%%% sazbu kódu, resp. výstupů z počítačových programů 
%%%%% ------------------------------------------------------------
\DefineVerbatimEnvironment{PCinout}{Verbatim}{fontsize=\small, frame=single}


%%%%% Další příkazy, které mohou zjednodušit tvorbu práce (často se
%%%%% vyskytující symboly atd.) 
%%%%% * vše by mělo být uvedeno na jednom místě (zde) 
%%%%% * v hlavním textu by se již nemělo (až na výjimky) nikde
%%%%%   vyskytovat \newcommand apod. 
%%%%% * níže je uvedeno několik příkladů příkazů, jež jsou (resp.
%%%%%   jejich modifikace a rozšíření) 
%%%%%   užitečné při sazbě matematického textu
%%%%% --------------------------------------------------------------

%%% prostor reálných, resp. přirozených čísel
\newcommand{\R}{\mathbb{R}}
\newcommand{\N}{\mathbb{N}}

%%% užitečné operátory pro statistiku a pravděpodobnost
\DeclareMathOperator{\pr}{\textsf{P}}
\DeclareMathOperator{\E}{\textsf{E}\,}
\DeclareMathOperator{\Eb}{\mathbb{E}}
\DeclareMathOperator{\var}{\textrm{var}}
\DeclareMathOperator{\sd}{\textrm{sd}}
\DeclareMathOperator{\Tr}{Tr}


%%% pro definování výrazů
\newcommand{\eqdef}{
    \stackrel{\mathrm{df}}{=}}
%%% diferenciál
\newcommand{\dd}{
    \,\mathrm{d}}

%%% příkaz pro transpozici vektoru/matice
\newcommand{\T}[1]{#1^\top}        

%%% různé šikovné vychytávky pro matematiku
\newcommand{\goto}{\rightarrow}
\newcommand{\gotop}{\stackrel{P}{\longrightarrow}}
\newcommand{\maon}[1]{o(n^{#1})}
\newcommand{\abs}[1]{\left|{#1}\right|}
\newcommand{\dint}{\int_0^\tau\!\!\int_0^\tau}
\newcommand{\isqr}[1]{\frac{1}{\sqrt{#1}}}

%%% různé šikovné vychytávky pro tabulky
\newcommand{\pulrad}[1]{\raisebox{1.5ex}[0pt]{#1}}
\newcommand{\mc}[1]{\multicolumn{1}{c}{#1}}



%%%%% Hlavní část dokumentu
%%%%% ---------------------

\begin{document}

%%% Pro přehlednost je vhodné umístit jednotlivé kapitoly 
%%% do samostatných souborů. Nepotřebné kapitoly můžeme zakomentovat.

%%%
%%%  VZOR PRO VYTVOŘENÍ DIPLOMOVÉ PRÁCE 
%%%  
%%%  * soubor obsahující titulní stránku a další náležitosti
%%%  vyskytující se na začátku každé práce 
%%%
%%%  AUTOŘI:  Martin Mareš (mares@kam.mff.cuni.cz)
%%%           Arnošt Komárek (komarek@karlin.mff.cuni.cz), 2011
%%%           Michal Kulich (kulich@karlin.mff.cuni.cz), 2013
%%%
%%%  POSLEDNÍ ÚPRAVA: 20130315
%%%  
%%%  ===========================================================================

\pagestyle{empty}
\begin{center}

%%% Titulní strana
%%% Tato stránka se nepřekládá do slovenštiny!!

{\large Univerzita Karlova v Praze}

\medskip
{\large Matematicko-fyzikální fakulta}

\vfill
{\bfseries\Large DIPLOMOVÁ PRÁCE}

\vfill
\centerline{\mbox{\includegraphics[width=60mm]{\FIGDIR/mfflogo.eps}}}

\vfill
\vspace{5mm}

{\LARGE Jan Poslušný}

\vspace{15mm}

%%% Název práce  v češtině přesně podle zadání
{\LARGE\bfseries  Optimální řízení stochastických lineárních rovnic s gaussovským šumem}

\vfill

%%% Název katedry nebo ústavu, kde byla práce oficiálně zadána
%%% (dle Organizační struktury MFF UK) 
%%% viz http://www.mff.cuni.cz/toUTF8.cs/fakulta/struktura/sekcem.htm
Katedra pravděpodobnosti a~matematické statistiky
% Katedra algebry
% Katedra didaktiky matematiky
% Katedra matematické analýzy
% Katedra numerické matematiky
% Katedra pravděpodobnosti a~matematické statistiky
% Matematický ústav UK


\vfill

\begin{tabular}{rl}
Vedoucí diplomové práce: & prof. RNDr. Bohdan Maslowski, DrSc. \\   %% Jméno a příjmení s~tituly 
\noalign{\vspace{2mm}}
Studijní program: & Matematika\\
\noalign{\vspace{2mm}}
Studijní obor: & Finanční a pojistná matematika\\
%Studijní obor: & Finanční a pojistná matematika\\
%Studijní obor: & Matematická analýza\\
%Studijní obor: & Matematické metody informační bezpečnosti\\
%Studijní obor: & Matematické modelování ve fyzice a technice\\
%Studijní obor: & Matematické struktury\\
%Studijní obor: & Numerická a výpočtová matematika\\
%Studijní obor: & Pravděpodobnost, matematická statistika a ekonometrie\\
%Studijní obor: & Učitelství matematiky-deskriptivní geometrie pro střední školy\\
%Studijní obor: & Učitelství matematiky-informatiky pro střední školy\\
\end{tabular}

\vfill

% Zde doplňte rok
Praha 2014

\end{center}




%%% Následuje vevázaný list -- kopie podepsaného "Zadání diplomové práce".
%%% Toto zadání NENÍ součástí elektronické verze práce, nescanovat.



\newpage
\openright

%%% Na tomto místě mohou být napsána případná poděkování (vedoucímu práce,
%%% konzultantovi, tomu, kdo zapůjčil software, literaturu apod.)
\noindent
Poděkování (nepovinné).




\newpage
%%% Strana s čestným prohlášením k diplomové práci
%%% Čestné prohlášení se nepřekládá do slovenštiny
\vspace*{\stretch{8}}

\noindent
Prohlašuji, že jsem tuto diplomovou práci vypracoval samostatně a~výhradně
s~použitím citovaných pramenů, literatury a~dalších odborných zdrojů.

\medskip\noindent
Beru na~vědomí, že se na moji práci vztahují práva a~povinnosti vyplývající
ze~zákona č.~121/2000 Sb., autorského zákona v~platném znění, zejména skutečnost,
že Univerzita Karlova v~Praze má právo na~uzavření licenční smlouvy o~užití této
práce jako školního díla podle \S60 odst.~1 autorského zákona.

\vspace{18mm}
%%% Před odevzdáním nezapomeňte každý výtisk podepsat
\noindent
V \makebox[4cm]{\dotfill} dne \makebox[2.5cm]{\dotfill}
\hspace*{\fill}
Podpis autora
\hspace*{\fill}

\vspace*{\stretch{1}}




\newpage
%%% Abstrakty v jazyce českém a anglickém

\vbox to 0.5\vsize{
\setlength\parindent{0mm}
\setlength\parskip{5mm}

Název práce:
Optimální řízení stochastických lineárních rovnic s gaussovským šumem

Autor:
Bc. Jan Poslušný

Katedra:  
Katedra pravděpodobnosti a~matematické statistiky

%%% (dle Organizační struktury MFF UK) 
%%% viz http://www.mff.cuni.cz/toUTF8.cs/fakulta/struktura/sekcem.htm
% Katedra algebry
% Katedra didaktiky matematiky
% Katedra matematické analýzy
% Katedra numerické matematiky
% Katedra pravděpodobnosti a~matematické statistiky
% Matematický ústav UK

Vedoucí diplomové práce:
prof. RNDr. Bohdan Maslowski, DrSc., Katedra pravděpodobnosti a~matematické statistiky
%%% pracoviště dle Organizační struktury MFF UK
%%% viz http://www.mff.cuni.cz/toUTF8.cs/fakulta/struktura/sekcem.htm
%%% případně plný název pracoviště mimo MFF UK
% Katedra algebry
% Katedra didaktiky matematiky
% Katedra matematické analýzy
% Katedra numerické matematiky
% Katedra pravděpodobnosti a~matematické statistiky
% Matematický ústav UK


Abstrakt:
Český abstrakt v rozsahu 80\,--\,200 slov. Nejedná se o~opis zadání diplomové práce

Klíčová slova:
optimální řízení, gaussovský šum, frakcionální Brownův pohyb

\vss}

\nobreak\vbox to 0.49\vsize{
\setlength\parindent{0mm}
\setlength\parskip{5mm}

Title:
Optimal Control of Stochastic Linear Equations with Fractional Noise

Author:
Bc. Jan Poslušný

Department:
Department of Probability and Mathematical Statistics 
%%% dle Organizační struktury MFF UK v angličtině
%%% viz http://www.mff.cuni.cz/toUTF8.en/fakulta/struktura/sekcem.htm
% Department of Algebra
% Department of Mathematics Education
% Department of Mathematical Analysis
% Department of Numerical Mathematics
% Department of Probability and Mathematical Statistics
% Mathematical Institute of Charles University

Supervisor:
prof. RNDr. Bohdan Maslowski, DrSc., Department of Probability and Mathematical Statistics
%%% dle Organizační struktury MFF UK v angličtině
%%% viz http://www.mff.cuni.cz/toUTF8.en/fakulta/struktura/sekcem.htm
%%% případně plný název pracoviště mimo MFF UK přeložený do angličtiny
% Department of Algebra
% Department of Mathematics Education
% Department of Mathematical Analysis
% Department of Numerical Mathematics
% Department of Probability and Mathematical Statistics
% Mathematical Institute of Charles University

Abstract:
Anglický abstrakt v~rozsahu 80\,--\,200 slov. Nejedná se o~překlad
zadání diplomové práce.

Keywords:
Optimal Control, Gaussian Noise, Fractional Brownian Motion
\vss}



% \newpage
%%% Slovenský abstrakt; tato strana se vkládá pouze do prací psaných ve
%%% slovenštině

% \vbox to 0.5\vsize{
% \setlength\parindent{0mm}
% \setlength\parskip{5mm}

% Názov práce:
% Názov práce preložený do slovenčiny 
% 
% Autor:
% Meno a priezvisko autora
% 
% Katedra:  
% Název katedry či ústavu, kde byla práce oficiálně zadána
%%% Název katedry dle Organizační struktury MFF UK
%%% viz http://www.mff.cuni.cz/toUTF8.cs/fakulta/struktura/
%%% Nepřekládat do slovenštiny!!!
% Katedra algebry
% Katedra didaktiky matematiky
% Katedra matematické analýzy
% Katedra numerické matematiky
% Katedra pravděpodobnosti a~matematické statistiky
% Matematický ústav UK

% Vedúci diplomovej práce:
% RNDr. Jméno Vedoucí, Ph.D., pracoviště
%%% dle Organizační struktury MFF UK
%%% případně plný název pracoviště mimo MFF UK
%%% Pracoviště nepřekládat do slovenštiny!!!
% Katedra algebry
% Katedra didaktiky matematiky
% Katedra matematické analýzy
% Katedra numerické matematiky
% Katedra pravděpodobnosti a~matematické statistiky
% Matematický ústav UK

% Abstrakt:
% Slovenský abstrakt v rozsahu 80\,--\,200 slov. Nejedná sa o preklad
% zadania diplomovej práce. Táto stránka sa vkladá iba do slovenských
% prác.

% Kľúčové slová:
% 3 až 5 kľúčových slov vo slovenčině

% \vss}



\newpage
\openright

%%% Strana s automaticky generovaným obsahem diplomové práce. U matematických
%%% prací je přípustné, aby případný seznam tabulek a zkratek, existují-li, byl umístěn
%%% na začátku práce, místo na jejím konci.

\pagestyle{plain}
\setcounter{page}{1}

\tableofcontents

%%% Změny se v automaticky generovaném obsahu projeví až po druhém
%%% zpracování zdrojového souboru (při prvním zpracování se pouze
%%% zapíšou do .toc souboru) 



%%%
%%%  VZOR PRO VYTVOŘENÍ DIPLOMOVÉ PRÁCE 
%%%  
%%%  * soubor obsahující fiktivní první kapitolu
%%%
%%%  AUTOŘI:  Arnošt Komárek (komarek@karlin.mff.cuni.cz), 2011
%%%           Michal Kulich (kulich@karlin.mff.cuni.cz), 2013
%%%
%%%  POSLEDNÍ ÚPRAVA: 20130315
%%%  
%%% =========================================================================

\chapter*{Úvod}
\addcontentsline{toc}{chapter}{Úvod}

\epigraph{\textit{``Begin at the beginning," the King said gravely, ``and go on
till you come to the end: then stop."}}{--- \textup{Lewis Carroll},\\ \textit{
Alice in Wonderland}}

Zde v bodech je shrnutí plánovaného členění této práce:
\begin{itemize}
    \item V první kapitole budou zavedeny pojmy, které se nadále budou používat
        v textu.
    \item Ve druhé kapitole budou shrnuty výsledky, ke kterým se již došlo v
        oblasti optimálního řízení rovnic s gaussovským šumem, se zaměřením na
        frakcionální Brownův pohyb.
    \item Ve třetí kapitole bude prostor pro případné důležité příklady a
        případné využití optimálního řízení ve finanční matematice.
    \item Čtvrtá kapitola by mohla nabídnout případné numerické porovnání výsledků
        dosahovaných při použití modelů s Brownovým pohybem a frakcionálním
        Brownovým pohybem a empirickými daty.
\end{itemize}

%Frakcionální Brownův pohyb představuje rodinu centrovaných gaussovských
%procesů, která je indexovaná Hurstovým parametrem $H\in\left(0,1\right)$.
%\vspace*{\baselineskip}

% \section*{Povídání}
%\addcontentsline{toc}{section}{Povídání}

%Řízení lineárních stochastických systémů s bílým šumem a hodnotovou funkcí
%kvadratickou v stavové proměnné (což se často nazývá LQG ({\em linear-quadratic
%Gaussian}) control problem) je pravděpodobně nejznámějším problémem
%stochastického optimálního řízení ve spojitém čase.

%Nyní bych rád popsal stávající situaci výzkumu:

Kleptsyna, LeBreton a Viot v pracích
\cite{kleptsyna2003linear,kleptsyna2005infinite} obdrželi optimální řízení pro
skalární stochastický systém s {\em fBm} pro $H\in\left(\frac{1}{2},1\right)$.
Pro případ částečné pozorovatelnosti navazují předem zmínění autoři prací
\cite{kleptsyna2008separation}. Hu a Zhou \cite{hu2005stochastic} odvodili
optimální řízení pro skalární bilineární systém s {\em fBm} pro
$H\in\left(\frac{1}{2},1\right)$ za podmínky, že takovéto řízení je markovské.
Duncan a Pasik-Duncan popsali některé speciální případy vícerozměrného systému
v \cite{duncan2009control} a \cite{duncan2010stochastic}. Optimální řízení
linearních rovnic v Hilbertově prostoru s {\em fBm} pro
$H\in\left(\frac{1}{2},1\right)$ a kvadratický hodnotový funkcionál explicitně
vyřešili Duncan, Maslowski a Pasik-Duncan v \cite{duncan2012linear}. Dále
Duncan a Pasik-Duncan v \cite{duncan2013linear} se zabývali optimálním řízením
lineárních stochastických systémů s libovolným centrovaným, kvadraticky
integrovatelným stochastickým procesem se spojitými trajektoriemi a hodnotovým
funkcionálem kvadratickým ve stavové proměnné, s explicitím vyjádřením případu
{\em fBm}, kde využili zobecněné metody doplnění na čtverec.


%%%%% ==============================================================================

%%%
%%%  VZOR PRO VYTVOŘENÍ DIPLOMOVÉ PRÁCE 
%%%  
%%%  * soubor obsahující fiktivní první kapitolu
%%%
%%%  AUTOŘI:  Arnošt Komárek (komarek@karlin.mff.cuni.cz), 2011
%%%           Michal Kulich (kulich@karlin.mff.cuni.cz), 2013
%%%
%%%  POSLEDNÍ ÚPRAVA: 20130315
%%%  
%%%  ===========================================================================

\chapter{Základní pojmy}

Nechť $\left( \Omega, \mathscr{F}, \mathbb{P} \right)$ je úplný
pravděpodobnostní prostor. Více na thema frakcionálního Brownova pohybu např.
\cite{nourdin2013selected}.


\begin{definice}\label{def:fbm}
    Standardním frakcionálním Brownovým pohybem (\vspace{2pt}fBm) indexovaný
    Hurstovým parametrem $H\in\left(0,1\right)$ nazveme spojitý centrovaný
    gaussovský proces $\left( B^H (t),t\geq 0\right)$ s kovarianční funkcí
    \begin{align*}
      \mathbb{E}& \left[ B^H (s) \, B^H (t) \right] =
      \frac{1}{2}\left( t^{2H}+s^{2H}-|t-s|^{2H} \right).
    \end{align*}
\end{definice}


\begin{definice}
    Náhodný proces $\left(B^H(t),t\geq0\right)$ nazveme cylindrickým frakcionálním Brownovým
    pohybem s Hurstovým parametrem $H\in\left(1/2,1\right)$ na prostoru $V$, definovaný na
    filtrovaném pravděpodobnostním prostoru $\left( \Omega, \mathscr{F}, \left(
    \mathscr{F}_t\right)_{t\geq0},\mathbb{P} \right)$, pokud
    
    $$ B^H(t) = \sum_{i=1}^\infty e_i \sqrt{\lambda_i}\, \beta_i (t), \qquad
    t>0,$$
    kde $\left\{ e_i, i\in\mathbb{N} \right\}$ je úplná ortonormální báze
    prostoru $V$, $\left( \beta_i(t),i\in\mathbb{N},t\geq0 \right)$ je rodina
    (stochasticky) nezávislých, reálných standardních frakcionálních
    Brownových pohybů s pevně zvoleným Hurstovým parametrem $H\in\left(
    {\frac{1}{2},1}\right)$ a $\lambda\geq0$, $\left(
    \lambda_i,i\in\mathbb{N} \right)$ je omezená poslopnoust v
    $\mathbb{R}_+$.
\end{definice}


\begin{definice}
    Řekneme, že filtrace $\left( \mathscr{F}_t,t\geq0 \right)$ splňuje obvyklé podmínky, pokud
    \begin{enumerate}[(i)]
        \item $\mathscr{F}_t$ je $\mathbb{P}$-úplná,
        \item $\mathscr{F}_t = \bigcap_{s>t} \mathscr{F}_s$  (zprava spojitá).
    \end{enumerate}
\end{definice}


\begin{definice}
    Inkrementální kovariance $\tilde{Q}$ cylindrického fBm $\left(B^H(t),t\geq0\right)$
    je definovaná vztahem
    $$ \tilde{Q}e_n = \lambda_n e_n, \qquad n\in\mathbb{N}. $$ 
\end{definice}


\begin{definice}[\cite{duncan2013linear}]\label{def:fracint}
    Nechť $\alpha\in \left( 0,1 \right)$ je pevné. Potom levostranný, popř.
    pravostranný frakcionální (Riemann--Liouvilleův) integrál pro
    $\varphi\in L^1 \left( \left[ 0,T \right] \right)$ definovaný pro skoro
    všechna $t\in\left[ 0,T \right]$ je dán předpisem
        
        $$\left( I_{0_+}^\alpha \varphi \right) \left( t \right) \eqdef
        \frac{1}{\Gamma(\alpha)}\int_{0}^{t} \left(t-s\right)^{\alpha-1} \varphi(s)
        \dd s,$$
    popř.
        $$\left( I_{T_-}^\alpha \varphi \right) \left( t \right) \eqdef
        \frac{1}{\Gamma(\alpha)}\int_{t}^{T} \left(s-t\right)^{\alpha-1} \varphi(s)
        \dd s,$$
        kde $\Gamma(\alpha)=\int_0^\infty x^{\alpha-1}e^{-x}\dd x$ je gama funkce.
\end{definice}


\begin{definice}[\cite{duncan2013linear}]\label{def:fracder}
    Nechť $\alpha\in \left( 0,1 \right)$ je pevné. Potom levostranná, popř.
    pravostranná frakcionální (Riemann--Liouvilleva) derivace pro
    $\varphi\in L^1 \left( \left[ 0,T \right] \right)$ definovaná pro skoro
    všechna $t\in\left[ 0,T \right]$ je dána předpisem
        
        $$\left( D_{0_+}^\alpha \varphi \right) \left( t \right) \eqdef
        \frac{\mathrm{d}}{\mathrm{d}t}\left( I_{0_+}^{1-\alpha} \varphi \right) \left( t \right) ,$$
    popř.
        $$\left( D_{T_-}^\alpha \varphi \right) \left( t \right) \eqdef
        \frac{\mathrm{d}}{\mathrm{d}t}\left( I_{T_-}^{1-\alpha} \varphi \right) \left( t \right) ,$$
    což ve Weylově reprezentaci dostáváme
        %$$\left( D_{0_+}^\alpha \varphi \right) \left( t \right) =
        %\frac{1}{\Gamma(1-\alpha)}\left[ \frac{\varphi(t)}{t^\alpha} + \alpha
            %\int_{0}^{t} \frac{\varphi(t)-\varphi(s)}{\left(t-s\right)^{\alpha+1}} 
        %\dd s, \right],$$
    %popř.
        %$$\left( D_{T_-}^\alpha \varphi \right) \left( t \right) =
        %\frac{1}{\Gamma(1-\alpha)}\left[ \frac{\varphi(t)}{\left( T-t
        %\right)^\alpha} + \alpha \int_{t}^{T} \frac{\varphi(s)-\varphi(t)}{\left(s-t\right)^{\alpha+1}} 
        %\dd s \right].$$
\begin{align*}
    \left( D_{0_+}^\alpha \varphi \right) \left( t \right) &=
        \frac{1}{\Gamma(1-\alpha)}\left[ \frac{\varphi(t)}{t^\alpha} + \alpha
        \int_{0}^{t} \frac{\varphi(t)-\varphi(s)}{\left(t-s\right)^{\alpha+1}} 
        \dd s, \right],\\
    \intertext{popř.}
    \left( D_{T_-}^\alpha \varphi \right) \left( t \right) &=
        \frac{1}{\Gamma(1-\alpha)}\left[ \frac{\varphi(t)}{\left( T-t
        \right)^\alpha} + \alpha \int_{t}^{T} \frac{\varphi(s)-\varphi(t)}{\left(s-t\right)^{\alpha+1}} 
        \dd s \right].
\end{align*}
\end{definice}


%\begin{definice}
    %Formální derivaci (\vspace{2pt}fBm) nazýváme frakcionálním gaussovským šumem (fGn).
%\end{definice}


\begin{definice}
    Prostorem $L^2_H \left( \left[ 0,T \right] \right)$ nazveme hilbertův
    prostor, kde $f,g\in L^2_H$, pokud $\langle f,f \rangle_H < \infty$ a
    $\langle g,g \rangle_H < \infty$ a skalární součin je dán jako

    $$ \langle f,g \rangle_H = \rho (H) \int_0^T u_{\frac{1}{2}-H}^2 (r)
    \left(I_{T^-}^{\left(H-\frac{1}{2}\right)} u_{H-\frac{1}{2}}\; f \right) (r)
    \left(I_{T^-}^{\left(H-\frac{1}{2}\right)} u_{H-\frac{1}{2}}\; g \right)
    (r) \dd r, $$
    kde 
    $$ \rho (H) = \frac{2H \, \Gamma (H+\frac{1}{2}) \, \Gamma(\frac{3}{2}-H)}
    {\Gamma (2-2H)}$$
    a $u_a (s) = s^a$, $a>0$, $s>0$.

\end{definice}


\begin{tvrz}[\cite{duncan2006prediction}]
    Nechť $f\in L^2_H \left( \left[ 0,T \right] \right)$. Potom
        $$\mathbb{E}\left[\int_s^t f \dd B^H \;|\; B^H (r), r\in \left(s,t\right)\right]
        = \int_0^s u_{\frac{1}{2}-H} \left(I_{s^-}^{-\left(H-\frac{1}{2}\right)} 
        \left(I_{t^-}^{\left(H-\frac{1}{2}\right)} u_{H-\frac{1}{2}}\; f
        \right) \right) \dd B^H, $$
    kde $u_a (s) = s^a$, $a>0$, $s>0$.
\end{tvrz}


%%%  ===========================================================================

%%%
%%%  VZOR PRO VYTVOŘENÍ DIPLOMOVÉ PRÁCE 
%%%  
%%%  * soubor obsahující fiktivní třetí kapitolu
%%%
%%%  AUTOŘI:  Arnošt Komárek (komarek@karlin.mff.cuni.cz), 2011
%%%           Michal Kulich (kulich@karlin.mff.cuni.cz), 2013
%%%
%%%  POSLEDNÍ ÚPRAVA: 20130315
%%%  
%%%  ===========================================================================

\chapter{Stochastické optimální řízení}


\section{Lineární kvadratické řízení (LQC)}
    \cite{duncan2012linear}
    Nechť $U,V$ a $\mathscr{H}$ jsou reálné Hilbertovy prostory a uvažujme
    stavovou rovnici
    \begin{align}
        \dd X(t) &= \left( A X(t) + B u(t) \right)\!\dd t + C \dd B^H(t)
        \label{eq:stateeq}\\
        X_0 &= x,
    \end{align}
    v prostoru $\mathscr{H}$, kde $t\geq0$, $x\in\mathscr{H}$, $A : D_A \subset
    \mathscr{H}\rightarrow\mathscr{H}$ je lineární, (obecně) neomezený
    operátor, jenž je infinitesimální generátor silně spojité semigrupy
    $\left(S(t),t\geq0\right)$, a dále $\left(B^H(t),t\geq0\right)$ je
    cylindrický frakcionální Brownův pohyb s Hurstovým parametrem 
    $H\in\left({\frac{1}{2},1}\right)$ na prostoru $V$, definovaný na
    filtrovaném pravděpodobnostním prostoru $\left( \Omega, \mathscr{F}, \left(
    \mathscr{F}(t),t\geq0\right),\mathbb{P} \right).$
    
    Můžeme předpokládat, že filtrace $\left( \mathscr{F}(t),
    t\geq0 \right)$ splňuje tzv. obvyklé podmínky. V případě zde uvažovaného
    problému optimálního řízení je přirozené předpokládat, že $\left( \mathscr{F}(t),
    t\geq0 \right)$ je $\mathbb{P}$-zúplněním $\left( \sigma \left( B(s), s\leq
    t \right),t\geq0 \right)$.
\\

    Nyní uveďme některé předpoklady:
    \begin{enumerate}[({A}1)]
        \item Nechť je jedna z následujících dvou podmínek splněna pro $B$ a $C$ v
            \eqref{eq:stateeq}:
            \begin{enumerate}[(a)]
                \item $B\in\mathscr{L}\left( U,\mathscr{H} \right)$,
                    $C\in\mathscr{L}\left( V,\mathscr{H} \right)$, kde
                    $U = \left( U,\langle \cdot,\cdot \rangle_U,|\cdot|_U
                    \right)$ je Hilbertův prostor (stavový prostor of
                    controls).
                \item $\left(S(t),t\geq0\right)$ je analytickou semigrupou a
                    existují konstanty $\alpha\in\left( 0,1 \right)$ a
                $\beta\in\left( 0,1 \right]$ takové, že
                $B\in\mathscr{L}\left(U, D_A^{\alpha-1} \right)$ a
                $C\in\mathscr{L}\left(U, D_A^{\beta-1} \right)$.
            \end{enumerate}
        \item Nechť $\mathscr{U}$ je rodina admissible controls, pak pro
            $u\in\mathscr{U} := L_\mathscr{F}^p=L_\mathscr{F}^p\left( \left(
            0,T \right)\times \Omega;U \right)$, kde $p > \frac{1}{a}$,
            $p\geq2$ je pevné a $L_\mathscr{F}^p$ označuje uzavřený lineární
            podprostor všech $\mathscr{F}(t)$-progresivně měřitelných procesů v
            $L^p \left( \left( 0,T \right)\times \Omega;U \right)$. Pokud
            $B\in\mathscr{L}\left( U,\mathscr{H} \right)$, potom stačí, aby
            $p\geq2$.
        \item Předpokládejme, že existuje $T_0>0$ a $\eta>0$ takové, že
            $$ \int_0^T \int_0^T r^{-\eta}s^{-\eta}
            |S(r)C\tilde{Q}^{\frac{1}{2}}|_{\mathscr{L}_2\left(
            V,\mathscr{H} \right)}|S(s)C\tilde{Q}^{\frac{1}{2}}|_{\mathscr{L}_2\left(
            V,\mathscr{H} \right)} \phi_H(r-s)\dd r\!\dd s$$
            je konečný a kde $\phi_H(r):=H\left( 2H-1 \right)|r|^{2H-2}$.
        \item $Q,G\in\mathscr{L}(\mathscr{H})$, $Q\geq0$, $G\geq0$,
            $R\in\mathscr{L}(U)$, $R\geq0$, $Q$, $G$ a $R$ jsou
            samoadjungované.
        \item 
            \begin{enumerate}[(a)]
                \item $\Tr\tilde{Q}<\infty$.
                \item $\beta\geq\alpha>1-H$.
                \item Inverzní operátor k $R$ je omezený, tedy
                    $R^{-1}\in\mathscr{L}(U)$, a
                    $G\in\mathscr{L}(\mathscr{H},D_{A^*}^{\alpha-1})$.
            \end{enumerate}
    \end{enumerate}
    

\section{Stochastický princip maxima}

%%%
%%%  VZOR PRO VYTVOŘENÍ DIPLOMOVÉ PRÁCE 
%%%  
%%%  * soubor obsahující fiktivní druhou kapitolu
%%%
%%%  AUTOŘI:  Arnošt Komárek (komarek@karlin.mff.cuni.cz), 2011
%%%           Michal Kulich (kulich@karlin.mff.cuni.cz), 2013
%%%
%%%  POSLEDNÍ ÚPRAVA: 20130315
%%%  
%%%  ===========================================================================



%%% Literatura 
%%% Reference se hledají v souboru priklady_literatury.bib. Aby se
%%% vytvořil seznam literatury, je třeba ocitovat alespoň jednu
%%% referenci, zkompilovat tento soubor latexem, pak bibtexem a znovu
%%% latexem. Tím se vytvoří seznam použitých referencí
%%% (MgrPrace.bbl) a vloží se do práce na místě, kde se nachází příkaz
%%% \bibliography, tedy sem. 
%%% 
\bibliography{literatura}

%%% Obrázky v diplomové práci, existují-li.
%\listoffigures

%%% Tabulky v diplomové práci, existují-li.
%\listoftables

%%% Použité zkratky v diplomové práci, existují-li, včetně jejich vysvětlení.
%\chapter*{Seznam použitých zkratek}
%\addcontentsline{toc}{chapter}{Seznam použitých zkratek}


%%% Přílohy k diplomové práci, existují-li (různé dodatky jako výpisy programů,
%%% diagramy apod.). Každá příloha musí být alespoň jednou odkazována z vlastního
%%% textu práce. Přílohy se číslují.

%\chapter*{Přílohy}
%\addcontentsline{toc}{chapter}{Přílohy}



\end{document}

