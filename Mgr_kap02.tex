%%%
%%%  VZOR PRO VYTVOŘENÍ DIPLOMOVÉ PRÁCE 
%%%  
%%%  * soubor obsahující fiktivní třetí kapitolu
%%%
%%%  AUTOŘI:  Arnošt Komárek (komarek@karlin.mff.cuni.cz), 2011
%%%           Michal Kulich (kulich@karlin.mff.cuni.cz), 2013
%%%
%%%  POSLEDNÍ ÚPRAVA: 20130315
%%%  
%%%  ===========================================================================

\chapter{Stochastické optimální řízení}

%\epigraph{\textit{``Il faut réfléchir avant d'agir."}}{--- \textup{Proverbe français}}


\section{Lineární kvadratické řízení (LQC)}
    \cite{duncan2012linear}
    Nechť $U,V$ a $\mathscr{H}$ jsou reálné Hilbertovy prostory a uvažujme
    stavovou rovnici
    \begin{align}
        \dd X(t) &= A X(t)\dd t + B u(t)\dd t + C \dd B^H(t)
        \label{eq:stateeq}\\
        X_0 &= x,\nonumber
    \end{align}
    v prostoru $\mathscr{H}$, kde $t\geq0$, $x\in\mathscr{H}$, $A : D_A \subset
    \mathscr{H}\rightarrow\mathscr{H}$ je lineární, (obecně) neomezený
    operátor, jenž je infinitesimálním generátorem silně spojité semigrupy
    $\left(S(t),t\geq0\right)$, a dále $\left(B^H(t),t\geq0\right)$ je
    cylindrický frakcionální Brownův pohyb s Hurstovým parametrem 
    $H\in\left({\frac{1}{2},1}\right)$ na prostoru $V$, definovaný na
    filtrovaném pravděpodobnostním prostoru $\left( \Omega, \mathscr{F}, \left(
    \mathscr{F}(t),t\geq0\right),\mathbb{P} \right).$
    
    Můžeme předpokládat, že filtrace $\left( \mathscr{F}(t),
    t\geq0 \right)$ splňuje tzv. obvyklé podmínky. V případě zde uvažovaného
    problému optimálního řízení je přirozené předpokládat, že $\left( \mathscr{F}(t),
    t\geq0 \right)$ je $\mathbb{P}$-zúplněním $\left( \sigma \left( B(s), s\leq
    t \right),t\geq0 \right)$.
\\

    Nyní uveďme předpoklady:
    \begin{enumerate}[({A}1)]
        \item Nechť je pro $B$ a $C$ z \eqref{eq:stateeq} splněna jedna z
            následujících dvou podmínek:
            \begin{enumerate}[(a)]
                \item $B\in\mathscr{L}\left( U,\mathscr{H} \right)$,
                    $C\in\mathscr{L}\left( V,\mathscr{H} \right)$, kde
                    $U = \left( U,\langle \cdot,\cdot \rangle_U,|\cdot|_U
                    \right)$ je Hilbertův prostor (stavový prostor
                    regulací).
                \item $\left(S(t),t\geq0\right)$ je analytickou semigrupou a
                    existují konstanty $\alpha\in\left( 0,1 \right)$ a
                $\beta\in\left( 0,1 \right]$ takové, že
                $B\in\mathscr{L}\left(U, D_A^{\alpha-1} \right)$ a
                $C\in\mathscr{L}\left(U, D_A^{\beta-1} \right)$.
            \end{enumerate}
        \item Nechť $\mathscr{U}$ je rodina přípustných regulací, pak pro
            $u\in\mathscr{U} \eqdef L_\mathscr{F}^p=L_\mathscr{F}^p( (
            0,T )\times$\\$\Omega;U )$, kde $p > \frac{1}{a}$,
            $p\geq2$ je pevné a $L_\mathscr{F}^p$ označuje uzavřený lineární
            podprostor všech $\mathscr{F}(t)$-progresivně měřitelných procesů v
            $L^p \left( \left( 0,T \right)\times \Omega;U \right)$. Pokud
            $B\in\mathscr{L}\left( U,\mathscr{H} \right)$, potom stačí, aby
            $p\geq2$.
        \item Předpokládejme, že existuje $T_0>0$ a $\eta>0$ takové, že
            $$ \int_0^T \int_0^T r^{-\eta}s^{-\eta}
            |S(r)C\tilde{Q}^{\frac{1}{2}}|_{\mathscr{L}_2\left(
            V,\mathscr{H} \right)}|S(s)C\tilde{Q}^{\frac{1}{2}}|_{\mathscr{L}_2\left(
            V,\mathscr{H} \right)} \phi_H(r-s)\dd r\!\dd s$$
            je konečný a kde $\phi_H(r)\eqdef H\left( 2H-1 \right)|r|^{2H-2}$.
        \item $Q,G\in\mathscr{L}(\mathscr{H})$, $Q\geq0$, $G\geq0$,
            $R\in\mathscr{L}(U)$, $R\geq0$, $Q$, $G$ a $R$ jsou
            samoadjungované.
        \item 
            \begin{enumerate}[(a)]
                \item $\Tr\tilde{Q}<\infty$.
                \item $\beta\geq\alpha>1-H$.
                \item Inverzní operátor k $R$ je omezený, tedy
                    $R^{-1}\in\mathscr{L}(U)$, a
                    $G\in\mathscr{L}(\mathscr{H},D_{A^*}^{\alpha-1})$.
            \end{enumerate}
    \end{enumerate}

\begin{definice}
Hodnotový funkcionál je definovaný jako
\begin{equation}
    J (x,y) \eqdef \frac{1}{2}\mathbb{E}\int_0^T \big( \langle Q X_s,
    X_s \rangle_\mathscr{H} + \langle R u_s, u_s\rangle_U \big) \dd s +
    \frac{1}{2} \mathbb{E} \langle G X_T, X_T \rangle_\mathscr{H},
    \label{def:costfunctional}
\end{equation}
pro $x\in\mathscr{H}$ a $u\in U$, kde $Q$, $R$ a $G$ jsou lineární
operátory splňující podmínku (A4).
\end{definice}

Úlohou je minimalizovat hodnotový funkcionál $J(x,u)$, tedy
\begin{equation}
    \tilde{J}(x) \eqdef \inf_{u\in\mathscr{U}} J(x,u),
    \label{def:costfunctionalinfimum}
\end{equation}
a (pro dané $x\in\mathscr{H}$) nalézt optimální regulaci
$\hat{u}\in\mathscr{U}$, která dosahuje infima 
\eqref{def:costfunctionalinfimum}, tj. $J(x,\hat{u}) = \tilde{J}(x)$.

Formální Riccatiho rovnice je v tomto případě
\begin{align}
	\dot{V} &= A^*V + VA - VBR^{-1}B^{*}V + Q,
	\label{Riccati}\\
	V(0) &= G,
\end{align}
která nezahrnuje lineární poruchový operátor $C$, takže lze uplatnit 
dobře známý výsledek pro deterministický případ.

Nechť $\Sigma^+ = \left\{ V\in\mathscr{L}(\mathscr{H}),V=V^*;V\geq0 \right\}$.

\begin{veta}[(Existence a jednoznačnost řešení Riccatiho rovnice)]
	Za splnění předpokladů (A1), (A4) a (A5) existuje pro libovolný operátor $V_0\in\Sigma^+\cap
	\mathscr{L}(\mathscr{H}, D_{A^*}^{1-\alpha})$ jednoznačná operátorová funkce 
	$V\in C_s([0,T],\mathscr{L}(\mathscr{H}, D_{A^*}^{1-\alpha}))\cap C_s([0,T],\Sigma^+)$
	taková, že
	\begin{equation}
		V(t) = S^*(t) V_0 S(t) + \int_0^t S^*(t-s)\big(Q-\left( B^*V(s) \right)^* R^{-1}B^*V(s)
		\big) S(t-s) \dd s
		\label{V}
	\end{equation}
	pro všechna $t\in[0,T]$, či (ekvivalentně)
	\begin{multline}
		\frac{\dd}{\dd t} \langle V(t)\,x,y\rangle_\mathscr{H} =
                \langle V(t)x, Ay\rangle_\mathscr{H} + \langle Ax, V(t)y\rangle_\mathscr{H} +
                \langle Qx,y\rangle_\mathscr{H} \\
                - \langle R^{-1}B^*V(t)x,B^*V(t)y\rangle_U
		\label{V2}
        \end{multline}
        pro všechna $t\in\left( 0,T \right)$, $x,y\in D_A$ a $V(0)=V_0$.
\end{veta}

Zobrazení $V$ splňující \eqref{V} (nebo ekvivalentně \eqref{V2}) se nazývá
{\em slabé řešení} rovnice \eqref{Riccati} s $V_0=G$.

\begin{lemma}
    Pokud jsou předpoklady (A1) a (A3) -- (A5) splněny, je náhodný proces
    $(\varphi(t)$, $t\in[0,T])$, 
    \begin{equation}
        \varphi(t) \eqdef \int_t^T U_P(s,t)P(s)C\dd B^H(s),
        \label{phi}
    \end{equation}
    dobře definovaným centrovaným gaussovským procesem v $L^p\left( \Omega\times
    \left( 0,T \right), D_{A^*}^{1-\alpha} \right)$.
    \label{}
\end{lemma}

\begin{veta}[(Optimální neadaptivní řízení)]
    Nechť (A1) -- (A5) jsou splněny, nechť $x\in\mathscr{H}$ a $u\in U$; nechť
    $\varphi$ je dáno \eqref{phi} a nechť $(X_t, t\geq 0)$ je řešením rovnice řízení 
    \eqref{eq:stateeq}. Nechť $V = L^p \left( [0,T]\times\Omega, U \right)$ je lineární prostor 
    neadaptivních regulací. Optimální regulace $\bar{v}\in\mathscr{V}$, jež je
    řešením optimalizační úlohy \eqref{eq:stateeq} --
    \eqref{def:costfunctional}, kde množina regulací $\mathscr{U}$ je nahrazena
    množinou $\mathscr{V}$, je dána
    \begin{equation*}
        \bar{v} = -R^{-1} B^* \left( P(t)X(t) + \varphi(t) \right),
        \label{}
    \end{equation*}
    kde $(X_t, t\in[0,T])$ je řešením rovnice řízení \eqref{eq:stateeq} pro
    $u=\bar{v}$.

    Optimální funkcionál $\tilde{J}$ je
    \begin{equation*}\label{}
    \begin{split}
        \tilde{J} &= \frac{1}{2} \langle P(0)x,x\rangle_\mathscr{H} -
        \frac{1}{2}\Eb\int_0^T\left| R^{-1}B^*\varphi(s) \right|_U^2\dd s \\
        &\qquad\qquad  + \int_0^T \int_0^s \Tr \left[ C^*P(s)U_P^*(s,r)C\tilde{Q}
        \right]\,\phi_H(r-s)\dd r\! \dd s\\
        &= \frac{1}{2} \langle P(0)x,x\rangle_\mathscr{H}\\
        &\quad - \frac{1}{2}\int_0^T \int_s^T\int_s^T \Tr \left[
        R^{-1} B^* U_P(r,s) P(r) C \tilde{Q} C^* P(q) U_P^*(q,s) R^{-1} B \right]\\
        &\qquad\qquad\qquad\qquad\qquad\qquad\qquad\qquad\qquad\qquad\qquad
        \phi_H(r-q) \dd q\! \dd r\! \dd s\\
        &\qquad\qquad + \int_0^T \int_0^s \Tr \left[ C^*P(s)U_P^*(s,r)C\tilde{Q}
        \right]\,\phi_H(r-s)\dd r\! \dd s.
    \end{split}
\end{equation*}
\end{veta}

\begin{veta}[(Optimální adaptivní řízení)]
    Jsou-li naplněny předpoklady (A1) -- (A5), potom existuje jednoznačná
    optimální regulace $\left( \bar{u}(t),t\in\left[ 0,T \right] \right)$ v
    $\mathscr{U}$ optimalizační úlohy \eqref{eq:stateeq} -- \eqref{def:costfunctional}, která je dána
    předpisem
    \begin{equation*}
        \bar{u}(t) = -R^{-1} B^* P(t) X(t) - R^{-1} B^* \psi(t)
        \label{optimalcontrol}
    \end{equation*}
    a kde
    \begin{equation*}
        \begin{split}
        \psi(t) &\eqdef \Eb \left[ \varphi(t)\,|\,\mathscr{F}_t \right] \\
        &= \int_0^t s^{-(H-\frac{1}{2})} \left( I_{t_-}^{-( H-\frac{1}{2})}
        \left( I_{t_-}^{(H-\frac{1}{2})} u_{(H-\frac{1}{2})} \left( U_P
        (\cdot,t) P(\cdot) C \right) \right) \right)\! (s) \dd B^H(s),
    \end{split}
    \end{equation*}
    kde $t\in \left[ 0,T \right]$, $u_a(s)=s^a I$ pro $s>0$, $a\in\R$ a
    $I_{t_-}^a$ značí levostranný Riemann-Liouvillův integrál (viz. Definici
    \eqref{def:fracint}).

    Optimální funkcionál je
    \begin{equation*}
    \begin{split}
        \tilde{J} = J(x,\bar{u}) &= \frac{1}{2} \langle P(0)x,x
        \rangle_\mathscr{H} -\frac{1}{2}\Eb\int_0^T \left| R^{-1} B^*
        \psi(s) \right|_U^2 \dd s \\
        &\quad + \int_0^T \int_0^s \Tr \left[ C^*P(s) U_P^*(s,r) C \tilde{Q} \right]
        \phi_H (r-s) \dd r\!\dd s\\
        &= \frac{1}{2} \langle P(0)x,x
        \rangle_\mathscr{H} -
        \int_0^T \int_0^s\int_0^s \Tr\left[ \eta(s,p)\tilde{Q}\eta(s,q) \right]
        \phi_H (p-q) \dd p\!\dd q\!\dd s \\
        &\quad + \int_0^T \int_0^s \Tr \left[ C^*
            P(s) U_P^*(s,r) C \tilde{Q} \right] \phi_H (r-s) \dd r\!\dd s,
        \label{}
    \end{split}
    \end{equation*}
    kde $\Eb[\varphi(t)|\mathscr{F}(t)]=\int_0^t \eta(t,s)\dd B^H(s)$.
\end{veta}

